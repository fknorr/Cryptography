\documentclass[a4paper]{scrreprt}
\usepackage[utf8]{inputenc}
\usepackage{amsmath}
\usepackage{amssymb}
\usepackage{amsopn}
\usepackage{enumitem}

\newcommand{\norm}[1]{\left\lVert #1 \right\rVert}
\newcommand\ol\overline
\newcommand\Z{\mathbb Z}
\newcommand\F{\mathbb F}
\DeclareMathOperator{\ess}{ess}
\DeclareMathOperator{\sgn}{sgn}
\newcommand{\df}{\mathrm{d}}
\setlength{\parindent}{0em}
\setlength{\parskip}{0.75em}
\renewcommand*{\arraystretch}{1.5}

\begin{document}

\section*{Exercise 1}

$\F_4 \;=\; \F_{2^2} \;=\; \F_2[x]/\langle x^2+x+1 \rangle \;=\; \{ f\in\F_2[x]:f\;=\;a_0+a_1x\}
        \;=\; \{0,1,x,1+x\}$

\vspace*{1em}
\textbf{Addition:}

\hspace*{1cm}\begin{tabular}{c||c|c|c|c}
    +  & $0$ & $1$ & $x$ & $x+1$ \\
    \hline
    \hline
    $0$ & $0$ & $1$ & $x$ & $x+1$ \\
    \hline
    $1$ & $1$ & $0$ & $x+1$ & $x$ \\
    \hline
    $x$ & $x$ & $x+1$ & $0$ & $1$ \\
    \hline
    $x+1$ & $x+1$ & $x$ & $1$ & $0$ \\
\end{tabular}

with
\[x+x \;=\; x\underbrace{(1+1)}_{=\;0} \;=\; 0\]
\[x+x+1 \;=\; (x+x)+1 \;=\; 0+1 \;=\; 1\]

\vspace*{1.5em}
\textbf{Multiplication:}

\hspace*{1cm}\begin{tabular}{c||c|c|c|c}
    $\cdot$  & \hspace*{0.9em}$0$\hspace*{0.9em} & $1$ & $x$ & $x+1$ \\
    \hline
    \hline
    $0$ & $0$ & $0$ & $0$ & $0$ \\
    \hline
    $1$ & $0$ & $1$ & $x$ & $x+1$ \\
    \hline
    $x$ & $0$ & $x$ & $x+1$ & $1$ \\
    \hline
    $x+1$ & $0$ & $x+1$ & $1$ & $x$ \\
\end{tabular}

with
\[x(x+1) \;=\; x^2+x \;=\; \underbrace{(x+1)}_{\;=\;x^2}+x \;=\; 1\]
\[(x+1)(x+1) \;=\; (x+1)^2 \;=\; x^2+\underbrace{2x}_{\;=\;0} + 1 \;=\; x^2+1 \;=\; (x+1)+1 \;=\; x\]

\textbf{Recall:} $\varphi:(A,\cdot,+) \rightarrow (B,\odot,\oplus)$ is an isomorphism if
\begin{enumerate}[label=(\roman*)]
    \item $\varphi$ is a homomorphism:
        \begin{itemize}
            \item $\varphi(1_A) = 1_B$
            \item $\forall a_1, a_2\in A: \varphi(a_1+a_2) = \varphi(a_1)\oplus\varphi(a_2)
                    \wedge \varphi(a_1\cdot a_2) = \varphi(a_1)\odot\varphi(a_2)$
        \end{itemize}
    \item $\varphi$ is bijective
\end{enumerate}

\begin{eqnarray*}
    \varphi: \Z_4 &\rightarrow& \F_4\\
    0 &\mapsto& \varphi(0)\;=\;0\\
    1 &\mapsto& \varphi(1)\;=\;1\\
    2 &\mapsto& \varphi(2)\;=\;x\\
    3 &\mapsto& \varphi(3)\;=\;x+1
\end{eqnarray*}

For $a_1=2$ and $a_2=2$, we have
\[\varphi(a_1\cdot a_2) \;=\; \varphi(2\cdot 2) \;=\; \varphi(4) \;=\; \varphi(0) \;=\; 0\]
\[\varphi(a_1)\cdot\varphi(a_2)\;=\;\varphi(2)\cdot\varphi(2)\;=\;x\cdot x\;=\;x^2\;=\;x+1\]

For $a_1=1$ and $a_2=3$:
\[\varphi(a_1+a_2)\;=\;\varphi(1+3)\;=\;\varphi(4)\;=\;\varphi(0)\;=\;0\]
\[\varphi(a_1)+\varphi(a_2)\;=\;\varphi(1)+\varphi(3)\;=\;1+x+1\;=\;x\]
\[0\,\neq\, x \;\Rightarrow\; \varphi(1+3)\,\neq\,\varphi(1)+\varphi(3)\]

\vspace*{1em}
\textbf{Addition:}

\hspace*{1cm}\begin{tabular}{c||c|c|c|c}
    +  & $0$ & $1$ & $2$ & $3$ \\
    \hline
    \hline
    $0$ & $0$ & $1$ & $2$ & $3$ \\
    \hline
    $1$ & $1$ & $2$ & $3$ & $0$ \\
    \hline
    $2$ & $2$ & $3$ & $0$ & $1$ \\
    \hline
    $3$ & $3$ & $0$ & $1$ & $2$ \\
\end{tabular}

\section*{Exercise 2}

\begin{eqnarray*}
\varphi(31)&=&31\left(1-\frac{1}{31}\right)\;=\;31-\frac{31}{31} \;=\; 30\\
\varphi(15)&=&\varphi(3\cdot5)\;=\;\varphi(3)\cdot\varphi(5)\;=\;
        3\left(1-\frac 1 3\right)\cdot 5\left(1-\frac 1 5\right)\;=\;8\\
    \varphi(16) &=& \varphi(2^4) = 2^4-2^3=8\\[0.3em]
    \varphi(49) &=& \varphi(7^2)=7^2-7^1=42\\[0.3em]
    \varphi(22785) &=& \varphi(5\cdot3\cdot7^2\cdot31)=4\cdot2\cdot42\cdot30=10080\\
\end{eqnarray*}
since with prime $p$ it holds
\begin{eqnarray*}
    \varphi(p^\alpha)&=&p^\alpha\cdot\left(1-\frac 1 p\right)=p^\alpha\left(\frac{p-1}p\right)\\
    &=& p^\alpha\cdot p^{-1}\cdot (p-1)\\
    &=& p^{\alpha-1}\cdot (p-1)=p^{\alpha}-p^{\alpha-1}
\end{eqnarray*}


\textbf{Hints:}\begin{itemize}
    \item If $N\;=\;p$ and $p$ is prime, $\varphi(p)\;=\;\left|\left\{a\in\Z_p:\gcd(a,p)\;=\;1\right\}\right|\;=\;|\Z_p^*|\;=\;p-1$
        \[p\cdot\prod_{i=1}^1\left(1-\frac 1 p\right)\;=\;p-1\]
    \item $N\;=\;p^\alpha$ ?
    \item $N=\prod_{i=1}^k p_i^{\alpha_i}$ ?
\end{itemize}


\section*{Exercise 3}

\begin{center}
    \renewcommand*{\arraystretch}{1}
    \begin{tabular}{ccccccccccccc}
        $m$&=&N&O&P&U&B&L&I&C&K&E&Y\\
        $x$&=&13&14&15&20&1&11&8&2&10&4&24\\
        $f_1(x)$&=&43&46&49&64&7&37&28&10&34&16&76\\
        &&R&U&X&M&H&L&C&K&I&Q&Y\\
        &&N&O&T&N&O&T&N&O&T&N&O\\
        $f_2(f_1(x))$&=&E&I&Q&Z&V&E&P&Y&B&D&M
    \end{tabular}
\end{center}


\end{document}

