\documentclass[a4paper]{scrreprt}
\usepackage[utf8]{inputenc}
\usepackage{amsmath}
\usepackage{amssymb}
\usepackage{amsopn}
\usepackage{enumitem}

\newcommand{\norm}[1]{\left\lVert #1 \right\rVert}
\newcommand\ol\overline
\newcommand\Z{\mathbb Z}
\DeclareMathOperator{\ess}{ess}
\DeclareMathOperator{\sgn}{sgn}
\newcommand{\df}{\mathrm{d}}
\setlength{\parindent}{0em}
\setlength{\parskip}{0.75em}

\begin{document}

\section*{Exercise 1}

\begin{enumerate}[label=\alph*)]
    \item Addition modulo 26.
        \[f:\Z_{26}\rightarrow\Z_{26},\;f_a(\omega)=\omega+a,\;a\in\Z_{26}\]
    \item\[c = m+3\; (\mathrm{mod}\; 26)\]
    \item \begin{flushleft}
            $m=$ \texttt{ZH KDYH WR ILQLVK IRXU DVVLJQPHQWV RQOB LQ RQH ZHHNHQG WKDW LV
WRR PXFK IRU RQH SHUVRQ SOHDVH FRPH WR WKH OLEUDUB DQG OHW XV GR
        LW WRJHWKHU VLQFH ZH DUH LQ WKH VDPH JURXS}
    \end{flushleft}
    \item \begin{flushleft}
        $c=$ \texttt{I AM SO SORRY DEAR ROJI I NEED TO ENJOY MY WEEKEND I AM ENJOYING THE
        SUMMER AT MUNICH WITH FLORIAN AND SEVEN OTHER FRIENDS IT WOULD BE NICE IF YOU JOIN
        US WE CAN DO ALL THE EXERCISE SHEETS ON THURSDAY EVENING SEE YOU SOON}
    \end{flushleft}
\end{enumerate}

\section*{Exercise 2}

English letters descending by frequency: \texttt{etaoinshrdlcumwfgypbvkjxqz}

\begin{enumerate}
    \item Assign the first few letters according to frequency.
    \item Swap letters in the translation map with ones of similar frequency if the result
        is unsound (e.g. \texttt{ta} $\rightarrow$ \texttt{to})
    \item Repeat until the message is decrypted.
\end{enumerate}

\section*{Exercise 3}

Read $a\mid b$ as ``$a$ divides $b$''.

\begin{itemize}
    \item \textbf{Reflexivity:} For $x\in\Z$, $N\mid x-x$ since $N\mid 0$ for any $N$.
    \item \textbf{Symmetry:} $x\sim_N y \Rightarrow N\mid x-y\Rightarrow\exists k\in\Z: x-y=k\cdot N
        \Rightarrow x-y=N\cdot(-k)\Rightarrow N\mid y-x\Rightarrow y\sim_N x$
    \item \textbf{Transitivity:} For $x\sim_N y$ and $x\sim_N z$, it holds that $x\sim_N z$:
        \[\left.\begin{array}{c}x\sim_N y\Rightarrow N\mid x-y\\
                y\sim_N z\Rightarrow N\mid y-z\end{array}\right\}\Rightarrow
            \left\{\begin{array}{c}\exists k_1\in\Z:x-y=N\cdot k_1\\
            \exists k_2\in\Z:y-z=N\cdot k_2\end{array}\right.\]
        \[x-y+y-z=N k_1 + N k_2\;\Rightarrow\;x-z=N(k_1+k_2)\;\Rightarrow
        \;N\mid x-z\;\Rightarrow\;x\sim_N z\]
\end{itemize}

\section*{Exercise 4}

\begin{enumerate}[label=\alph*)]
    \item $4536782793\;\mathrm{mod}\;9784537=6542162$
    \item \label{4b} ...
    \item Use the program from \ref{4b} to calculate
        \[\gcd(16534528044, 8332745927)=(43, 81440996, -161602003)\]
        The second and third coefficient are called $x$ and $y$, such that
        $\gcd(a,b)=x\cdot a+y\cdot b$
    \item $\det n\geq 2$. We want to show that \[n=\prod_{i=1}^kp_i^{\alpha_i}\] and $n$ is unique.
        \begin{itemize}
            \item For $n=2$, $n$ is a prime.
            \item For $n > 2$:
                \begin{enumerate}[label=(\roman*)]
                    \item If $n=p$ where p is a prime
                    \item If $n$ is not a prime and $\ast$ is true for some $2\leq m<n$.

                        Since $n$ is not prime, $n$ is compsite, i.e. $n=m_1\cdot m_2$ with
                        $2\leq m_1<n$ and $2\leq m_2<n$.

                        Since $m_1 <n$ and $m_2<n$, by induction hypothesis, we have
                        \[m_1=\prod_{i=1}^{k_1}p_i^{\alpha_i},\quad
                            m_2=\prod_{j=1}^{k_2}q_j^{\beta j}\]

                        $p_i, p_j$ prime. Thus \[n=\prod_{l=1}^tp_e^{\gamma_l}\]
                \end{enumerate}

            $n$ is uniquely decomposed: Suppose\[n=\prod_{i=1}^kp_i^{\alpha_i},\quad
                n=\prod_{j=1}^lq_j^{\beta_j}\]
            We find that $q_1=p_1$.
        \end{itemize}
\end{enumerate}

\section*{Exercise 4}

Recall: $f:A\rightarrow B$, $f$ is called \textit{injective} iff $f(a)=f(b)\;\Rightarrow\; a_1=a_2$.

\begin{enumerate}
    \item \begin{eqnarray*}
            f_1(a)=f_1(b) &\Rightarrow& \ol 5 a + \ol 7 = \ol 5 b + \ol 7\\
            &\Rightarrow& \ol 5 a = \ol 5 b \\
            &\Rightarrow& a = \ol 5\cdot \ol 5^{-1} b \\
            &\Rightarrow& a = \ol 5\cdot\ol 5 b = 1 b = b
        \end{eqnarray*}
            since $\ol 5^{-1} = \ol 5$, so $f_1$ is injective.
        $f_2(\ol 1) = \ol 7 = f_2(\ol 7)$, but $\ol 1 \neq \ol 7$, so $f_2$ is not injective

    \item $c=$ \texttt{CUEMBSU LB OPWWT}
\end{enumerate}

\end{document}

