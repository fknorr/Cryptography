\documentclass[a4paper]{scrreprt}
\usepackage[utf8]{inputenc}
\usepackage{amsmath}
\usepackage{amssymb}
\usepackage{amsopn}
\usepackage{enumitem}

\newcommand{\norm}[1]{\left\lVert #1 \right\rVert}
\DeclareMathOperator{\ess}{ess}
\DeclareMathOperator{\sgn}{sgn}
\newcommand{\df}{\mathrm{d}}
\setlength{\parindent}{0em}
\setlength{\parskip}{0.75em}

\begin{document}

\section*{Exercise 1}
\subsection*{Exercise 1.1}

Recall:
\[\mathbb Z_N = \{\overline 0, \overline 1, \hdots, \overline{N-1}\}\]
e.g.
\[\mathbb Z_7 = \{\overline 0, \overline 1, \overline 2, \overline 3, \overline 4,
\overline 5, \overline 6 \}\]

$\mathbb Z_N^\ast$ is the set of invertible elements in $\mathbb Z_N$:
\[\mathbb Z_N^\ast = \{a\in\mathbb Z_N:\exists b\in\mathbb Z_N: a\cdot b=1\}\]
e.g. for $N = 26$, $\overline 3\cdot \overline 9 = \overline 27=\overline 1$, so $3$ and $7$
are in $\mathbb Z_{26}^*$.

Instead of finding inverses for all elements of $\mathbb Z_N$, we can use the euler function
$\varphi(N)$:
\[\left|Z_{26}^*\right|\;=\;\left|\left\{a\in\mathbb Z_N:
    \gcd(a,N)=1\right\}\right|\;=\;\varphi(N)\]

\textbf{Definition} (Euler Function)\begin{itemize}
    \item If $N=p$, $p$ prime: $\varphi(p) = p-1$
    \item If $N=p^k$, $p$ prime: $\varphi(N) = p^k-p^{k-1}=p^k\left(1-\frac 1 p\right)$
    \item If $N=\prod_{i=1}^k p_i^{\alpha_i}$, $p_i$ prime:
        $\varphi(N)=N\prod_{i=1}^k\left(1-\frac 1 {p_i}\right)$
    \item If $N=p\cdot q$: $\varphi(N)=\varphi(p)\cdot\varphi(q)$
\end{itemize}

e.g.
\begin{eqnarray*}
    \varphi(7) &=& 7-1 = 6 = |\mathbb Z_7^*|\\
    \varphi(25) &=& 5^2-5 = 25- 5 = 20 = |\mathbb Z_{25}^*|\\
    \varphi(26) &=& \varphi(2\cdot 13) = \varphi(2)\cdot\varphi(13) (2- 1)\cdot(13-1) = 12 = |\mathbb Z_{25}^*|\\
\end{eqnarray*}

\subsection*{Exercise 1.2}
\[A=\begin{pmatrix}a&c\\b&d\end{pmatrix}\qquad \det(A)=ad-bc\]
    $A$ is invertible iff $\det(A) \neq 0$, i.e. the columns of $A$ are linearly independent:
For the columns
    \[C_1=\begin{pmatrix}a\\b\end{pmatrix}\qquad C_2=\begin{pmatrix}c\\d\end{pmatrix}\]
are independent, which means that there is no $\lambda$ such that $C_1 =\lambda C_2$.

We need to remove $C_2 = 0$ since it's always linearly dependent, so there are $p^2 - 1$ choices for $C_2$.
For $C_1$, we can choose one component freely and have one choice less for the second component as to avoid
linear dependence: There are $p^2-p$ choices.

\[\left|GL_2(\mathbb Z_p)\right|\;=\;(p^2-1)(p^2-p)\]

\subsection*{Exercise 1.3}

``$\Rightarrow$'' Suppose that $A\in\mathbb Z_N^{k\times k}$ is invertible. Show that $\det(A)\in\mathbb Z_N^*$.

If $A$ is invertible, there exists $B\in\mathbb Z_N^{k\times k}$ such that $A\cdot B = I$.
TODO

``$\Leftarrow$'' Suppose that $A\in\mathbb Z_N^{k\times k}$ such that $\det(A)\in\mathbb Z_N^*$.
\[\det(A)\in\mathbb Z_N^{k\times k}\;\Rightarrow\;\exists b\in\mathbb Z_N: \det(A)\cdot b=1\]
\[\exists M\in\mathbb Z_N^{k\times k}: \det(M) = b\]
\[\det(A)\cdot\det(M)\;=\;1\;=\;\det(I)\]
??? Instead prove indirectly by contradiction.

\subsection*{Exercise 1.4}

$|GL_2(\mathbb Z_{26})| = ?$ 


$\mathbb Z_{26}$ is isomorph to $\mathbb Z_2\times\mathbb Z_{13}$
\begin{eqnarray*}
    \mathbb Z_{26} &\approxeq& \mathbb Z_2\times\mathbb Z_{13}\\
    GL_2(\mathbb Z_{26}) &\approxeq& GL_2(\mathbb Z_2\times \mathbb Z_{13})
        \;\approxeq\;GL_2(\mathbb Z_2)\times GL_2(\mathbb Z_{13})\\
    |GL_2(\mathbb Z_{26})| &=& |GL_2(\mathbb Z_2)|\cdot|GL_2(\mathbb Z_{13})|
        \;=\;(2^2-1)(2^2-2)\cdot(13^2-1)(13^2-13)
\end{eqnarray*}

\section*{Exercise 2}

$M = \mathrm{NOMOREWAR}$

\subsection*{Exercise 2.1}

Affine map $f: \mathbb Z_{26}\rightarrow\mathbb Z_{26},\; x\mapsto\overline ax+\overline b$

\begin{itemize}
    \item N $=13$, $f(13) = \overline 7\cdot 13+\overline 5=\overline {96} = \overline{18}$, $18 \equiv $ S
    \item O $=14$, $\hdots$
\end{itemize}

Gives the ciphertext $T=$ YFRFANJLA.

\subsection*{Exercise 2.2}

\textbf{Encryption:} $g:\mathbb Z_{26}^2\rightarrow\mathbb Z_{26}^2,\; X\mapsto AX+B$.

\[X_1 = \begin{pmatrix}N\\O\end{pmatrix}=\begin{pmatrix}13\\14\end{pmatrix}\]
    \[f(X_1)=\begin{pmatrix}5&2\\1&1\end{pmatrix}\begin{pmatrix}13\\14\end{pmatrix}
        + \begin{pmatrix}3\\7\end{pmatrix}
            = \begin{pmatrix}\overline{5\cdot 13+2\cdot 14+3}\\\overline{1\cdot 13+1\cdot 14+7}\end{pmatrix}
                = \begin{pmatrix}\overline{18}\overline 7\end{pmatrix}
                    = \begin{pmatrix}\mathrm S\\\mathrm I\end{pmatrix}\]

If the length of $M$ is not divisible by $2$, pad the string with additional letters (e.g. NOMOREWARR)

Ciphertext $T=$ SINHSCJDSP.
\vspace{1em}

\textbf{Decryption:} $g^{-1}: \mathbb Z_{26}^2\rightarrow\mathbb Z_{26}^2,\; X\mapsto A^{-1}(X-B)$

\subsection*{Exercise 2.3}
Find $a, b$ such that $f(6) = 23$ and $f(15)=24$.

\[\left\{\begin{matrix}6a+b=23\\15a+b=24\end{matrix}\right.\;\Rightarrow\;a=3,b=5\]


\section*{Exercise 3}
\subsection*{Exercise 3i)}

\begin{samepage}
\textbf{Existance:} Suppose $\gcd(a-1, N)=1$ exists.\begin{eqnarray*}
    m = am+b &\Rightarrow& am-m+b = 0\\
    &\Rightarrow& (a-1)m+b=0\\
    &\Rightarrow& (a-1) m = -b\\
    &\Rightarrow& m = -b(a-1)^{-1}\\
\end{eqnarray*}
$m$ exists since $(a-1)^{-1}$ exists.
\end{samepage}

\textbf{Uniqueness:} ???

\subsection*{Exercise 3ii)}

Suppose $\gcd(m_1-m_2,N)=1$.
\[\left\{\begin{matrix}f(m_1)=c=1\\f(m_2)=c_2\end{matrix}\right.
    \;\Rightarrow\;\left\{\begin{matrix}am_1+b=c_1\quad (l_1)\\am_2+b=c_2\quad (l_1)\end{matrix}\right.\]

        $(l_1)-(l_2)$ gives $(m_1-m_2)a=c_1-c_2$ since $(m_1-m_2)$ exists (because $\gcd(m_1-m_2,N)=1$)

        We have: $a=(c_1-c_2)(m_1-m_2)^{-1}$ and $b=c_1-am_1$

\end{document}

