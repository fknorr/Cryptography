\documentclass[a4paper]{scrreprt}
\usepackage[utf8]{inputenc}
\usepackage{amsmath}
\usepackage{amssymb}
\usepackage{amsopn}
\usepackage{listings}
\usepackage{enumitem}

\newcommand{\norm}[1]{\left\lVert #1 \right\rVert}
\newcommand\ol\overline
\newcommand\N{\mathbb N}
\newcommand\Z{\mathbb Z}
\newcommand\F{\mathbb F}
\DeclareMathOperator{\ess}{ess}
\DeclareMathOperator{\sgn}{sgn}
\newcommand{\df}{\mathrm{d}}
\setlength{\parindent}{0em}
\setlength{\parskip}{0.75em}
\renewcommand*{\arraystretch}{1.0}
\lstset{basicstyle=\ttfamily,breaklines=true}

\begin{document}

\section*{Exercise 1}

$p$ is a prime, fix $g\in\F_p^\ast$:
\[f:(\Z_{p-1},+)\rightarrow(\F_p^\ast,\cdot),\;x\mapsto f(x)=g^x\]

\subsection*{Exercise 1.1}
Show that $f$ is a group homomorphism.

For $x,y\in\Z_{p-1}$:
\[f(x+y)=g^{x+y}=g^x\cdot g^y=f(x)\cdot f(y)\]

\subsection*{Exercise 1.2}
    $f$ is bijective $\Leftrightarrow\;\langle g\rangle=\F^\ast_p$.

$g$ is a generator of $\F_p^\ast$ if 
\[\F_p^\ast=\langle g\rangle:=\{g^i\mid i\in\{0,\hdots,p-1\}\}\]

``$\Rightarrow$'': $f$ si bijective, which means that 
$\forall y\in\F_p^\ast\exists!x\in\Z_p:y=g^x$, so by pluggin all elements from $\Z_p$ in to
$f$ we can generate all elements in $\F_p^\ast$. Therefore $g$ is a generator.

``$\Leftarrow$'' $\F_p^\ast=\{g^i\mid i\in\{0,\hdots,p-1\}\}$. Now we show that $f$ is bijective.
In our case, $f$ bijective $\Leftrightarrow f$ surjective $\Leftrightarrow f$ injective.

By looking at the definition of $f(i)=g^i$ and the definition of $\langle g\rangle$ we
immedialtely see that $f$ is surjective.


\section*{Exercise 2}
\subsection*{Exercise 2.1}
\texttt{checkPrime($n$)}\\
Input: an integer $n$\\
Output: \texttt{TRUE} if n is a prime or \texttt{FALSE} otherwise.

\begin{lstlisting}
If n == 1 Then Return FALSE
ElIf n == 2 Then Return TRUE
Else
    f = Floor(Sqrt(n))
    For k=2 To f Do
        If n Mod k == 0 Then
            Return FALSE
        EndIf
    EndFor
    Return True
EndIf
\end{lstlisting}

\subsection*{Exercise 2.2}

Prime numbers $< 100$: 2, 3, 5, 7, 11, 13, 17, 19, 23, 29, 31, 37, 41, 43, 47, 53, 59, 61, 67,
71, 73, 79, 83, 89, 97

\subsection*{Exercise 2.3}

\texttt{randomPrime()}\\
Input: -\\
Output: a prime number

\begin{lstlisting}
Repeat
    n = Random(100000)
    t = checkPrime(n)
Until t == TRUE
Return n
\end{lstlisting}


\section*{Exercise 3}

\subsection*{Exercise 3a)}

Dominant terms and big-O notaion.

\begin{enumerate}
    \item \(f_1=10^3n+5+0.0001n^3\):
        \[\lim_{n\rightarrow\infty}\frac{f_1}{n^3}\;=\;\lim_{n\rightarrow\infty}\frac{10^3n+5+0.0001n^3}{n^3}
        \;=\;\lim_{n\rightarrow\infty}\frac{10^3}{n^2}+\frac 5{n^3}+0.0001\;=\;0.0001\]
        $f\in\mathcal O(n^3)$ since the dominant term is $0.0001n^3$.

    \item \(f_2=30n+5n^{1.5}+2.5n^{1.71}\):

        By the same reasoning, $f\in\mathcal O(n^{1.71})$ by the dominant term $2.5n^{1.71}$.

    \item  \(f_3\in\mathcal O(n^2\log n)\)
    \item  \(f_4\in\mathcal O(n\log n)\)
    \item  \(f_5\in\mathcal O(\log n)\)
\end{enumerate}

\subsection*{Exercise 3b)}

\begin{enumerate}[label=(\roman*)]
    \item $g_1\in\mathcal O(g_2),\,g_2\in\mathcal O(g_3)$. Is $g_1\in\mathcal O(g_3)$?
        \[\forall n\in\N:\exists c>0:g_1(n)\leq c\cdot g_2(n)\]
        \[\forall n\in\N:\exists \tilde c>0:g_2(n)\leq \tilde c\cdot g_3(n)\]
        So we can concludde that
        \[\forall n\in\N:g_1(n)\leq c\cdot g_2(n)\leq c\cdot\tilde c\cdot g_3(n)\]
        Therefore $g_1\in\mathcal O(g_3)$.

    \item $g=5n+8n^2+0.1n^3\in\mathcal O(n^4)$?
        \[\lim_{n\rightarrow\infty}\frac{5n+8n^2+0.1n^3}{n^4}\;=\;0\]
        Therefore $g\in\mathcal O(n^4)$.

    \item $g=5n+8n^2+0.1n^3\in\mathcal o(n^3)$:
        \[\lim_{n\rightarrow\infty}\frac g{n^3}\;=\;0.1\;\neq\;0\]
        So $g\not\in\mathcal o(n^3)$.

    \item $g=5n+8n^2+0.1n^3\in\mathcal O(n^2\log n)$?
        \[\lim_{n\rightarrow\infty}\frac{n^3}{n^2\log n}\;=\;\lim_{n\rightarrow\infty}\frac n{\log n}
            \;=\;\lim_{n\rightarrow\infty} \frac 1{\frac 1 n}\;=\;\lim_{n\rightarrow\infty}n=\infty\]
        So $g\not\in\mathcal O(n^2\log n)$.
\end{enumerate}


\section*{Exercise 4}

Solve\[S_1:\left\{\begin{array}{c}x=3\mod 7\\x=7\mod 15\\x=9\mod 11\end{array}\right.\]

Chinese Remainder Theorem:
\[\gcd(7,15)=\gcd(7,11)=\gcd(15,11)=1\]
\[M=7\cdot15\cdot11\]
\[M_1=\frac M 7=15\cdot 11\]
\[M_2=\frac M {15}=7\cdot 11\]
\[M_3=\frac M {11}=7\cdot 15\]

\begin{eqnarray*}
    \gcd(M_1,7)=1 &\Rightarrow& \exists u,v:M_1\cdot u + 7 \cdot v=1\\
    &\Rightarrow& M_1\cdot u=1\mod 7\\
    &\Rightarrow& u=M_1^{-1}\mod 7=2
\end{eqnarray*}
\begin{eqnarray*}
    \gcd(M_2,15)=1 &\Rightarrow& \exists u_2:M_2\cdot u_2=1\mod 15\\
    &\Rightarrow& u_2=8=M_2^{-1}\mod 15
\end{eqnarray*}
\begin{eqnarray*}
    \gcd(M_3,11)=1 &\Rightarrow& \exists u_3:M_3\cdot u_3=1\mod 11\\
    &\Rightarrow& u_3=2=M_3^{-1}\mod 11
\end{eqnarray*}

Therefore
\[x=(3\cdot M_1\cdot u+7\cdot M_2\cdot u_2+9\cdot M_3\cdot u_3)\mod M\]
Hence $x=262$.

Likewise for $S_2$, $x=4959$.

\end{document}

