\documentclass[a4paper]{scrreprt}
\usepackage[utf8]{inputenc}
\usepackage{amsmath}
\usepackage{amssymb}
\usepackage{amsopn}
\usepackage{listings}
\usepackage{enumitem}

\newcommand{\norm}[1]{\left\lVert #1 \right\rVert}
\newcommand\ol\overline
\newcommand\N{\mathbb N}
\newcommand\Z{\mathbb Z}
\newcommand\F{\mathbb F}
\DeclareMathOperator{\ess}{ess}
\DeclareMathOperator{\sgn}{sgn}
\newcommand{\df}{\mathrm{d}}
\setlength{\parindent}{0em}
\setlength{\parskip}{0.75em}
\renewcommand*{\arraystretch}{1.5}

\lstset{basicstyle=\small\ttfamily}

\begin{document}

\section*{Exercise 1}

Let $M=4128768$.

\subsection*{Exercise 1.1}

For \[M=\prod_{i=1}^k p_i^{\alpha_i},\quad B=\hdots\], $M$ is $B$-smooth if $\forall i=1,\hdots,k,\;p_i\leq B$

\begin{enumerate}[label=(\alph*)]
    \item $B=4$: 
        \[p = u^{-u} = \frac 1 {u^u}\]
        \[u=\frac{\log M}{\log B}=\log_B(M)\]
        For $B=4$, $p=3.64\cdot 10^{-12}$
    \item $B=63$, $p=0,0083$
    \item $B=126$, $p=0,027$
    \item $B=589824$, $p=0.85$
\end{enumerate}

\subsection*{Exercise 1.2}

Suppose \lstinline{primes(k)} gives us the $k$-th prime number.

\subsubsection*{Implementation 1}
\begin{lstlisting}
    funciton smooth(n, B)
    input: n the number to check
           B from B-smooth
    output: True, if n is B-smooth
            False otherwise
    begin
        for k = 0 to infinity
            p = primes(k)
            if p > B and p|n
                return False
            if p > n
                return True
    end
\end{lstlisting}

\subsubsection*{Implementation 2}
\begin{lstlisting}
    div = [1]
    m = ceil(sqrt(n))

    for i = 1 to m do
        if n mod i == 0 then
            if i > B then
                return False
            else
                append i to div
            end
        end
    end

    if div == [1] then
        if n > B then
            return False
        else
            return True
        end
    else
        return True
    end
\end{lstlisting}


\subsection*{Exercise 1.3}

$n = M = 4128768,\;B = 4$

Prime factorization of $M$: Largest prime factor is $7$

\section*{Exercise 2}

Find $x$ such that $7^x = 35\mod 101$.

\begin{enumerate}
    \item Find a factor base: $\mathcal F := \{2,3,5,7\}$. 7 (the generator) should be part of $\mathcal F$.
    \item Find relations.
        \begin{eqnarray*}
            7^7&=&2\cdot3^2\cdot5\mod101 \\
            1&=&7^{-7}\cdot2^1\cdot3^2\cdot5^1\mod 101 \\
            \log1&=&\log(7^{-7}\cdot2^1\cdot3^2\cdot5^1) \\
            0 &=& \log_77^{-7}+\log_72^1+\log_73^2+\log_75^1 \\
            &=& -7\log_77+\log_72+2\log_73+\log_75 \\
            &=& -7 + x_2+2x_3+x_5\\
            7 &=& x_2+2x_3+x_5 
        \end{eqnarray*}

        ...

        \[\begin{pmatrix}-7 & 1 & 2 & 1\\-12 & 1 & 0\\-23 & 0 & 3 & 0\end{pmatrix}
            \begin{pmatrix}\log_77\\\log_72\\\log_73\\\log_75\end{pmatrix}
                = \begin{pmatrix}0\\0\\0\end{pmatrix}\]

        \[log_72=x_2=-11,\quad\log_73=x_3=41,\quad\log_75=x_5=36\]
    \item The last relation

        $h\cdot g=\prod_{p\in\mathcal F}p$, what is $g$?

        \[\cdot7^0=35=5\cdot7\mod 101\]
        \[35=5\cdot 7\]
        \[\log_735=\log_7(5\cdot7)=\log_75+\log_77 = 36=1=37=x\]

        ...
\end{enumerate}

\section*{Exercise 3}

\textit{Similar to last exercise sheet.}

\section*{Exercise 4}

Factorization: Search primes in the magnitude of $\sqrt{n}$ to find $p$ and $q$.

\section*{Exercise 5}

\begin{enumerate}
    \item Choose $e$ between ${64,9,15,229}$.

        $\gcd(e,\varphi(n))=1, \varphi(n)=2880=2^6\cdot3^2\cdot5$

        $e=229$
    \item $ed\equiv1\mod(\varphi(n))$
        
        $d=1069$
    \item $M=13, C\equiv M^e\mod3007\equiv2687\mod3007$
    \item $M\equiv C^d\mod3007\equiv13\mod 3007$
    \item $(p-1)(q-1)=\varphi(n)$

        $\varphi(n)=n-2a+1$
\end{enumerate}

\end{document}

