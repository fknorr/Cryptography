\documentclass[a4paper]{scrreprt}
\usepackage[utf8]{inputenc}
\usepackage{amsmath}
\usepackage{amssymb}
\usepackage{amsopn}
\usepackage{enumitem}

\newcommand{\norm}[1]{\left\lVert #1 \right\rVert}
\newcommand\ol\overline
\newcommand\N{\mathbb N}
\newcommand\Z{\mathbb Z}
\newcommand\F{\mathbb F}
\DeclareMathOperator{\ess}{ess}
\DeclareMathOperator{\sgn}{sgn}
\newcommand{\df}{\mathrm{d}}
\setlength{\parindent}{0em}
\setlength{\parskip}{0.75em}
\renewcommand*{\arraystretch}{1.5}

\begin{document}

\section*{Exercise 1}

$m_1=(7)$, $m_2=(7,1)$

\subsection*{Exercise 1.1}

Find all polynomials, then filter out the ones that are not irreducible.

There are 16 polynomials: Every P. has the form
\[x^4 + ax^3 + bx^2 + c^1 + d\]
For $a,b,c,d$ there are two possible values out of $\{\overline 0, \overline 1\}$,
so there are $2^4=16$ possibilies.

An irreducible polynomial in $\F_2[x]$ always has an odd number of terms, since a polynomial with
an even number of terms has a root.

In a table:
\begin{center}
    \renewcommand*{\arraystretch}{1}
    \begin{tabular}{ccccc|c}
        16 & 8 & 4 & 2 & 1 & \\
        $x^4$ &$x^3$ &$x^2$ & $x^1$ & $1$ & Decimal \\
        \hline
        1 & 0 & 0 & 0 & 0 & 16 \\
        1 & 1 & 1 & 0 & 0 & 28 \\
        1 & 1 & 0 & 1 & 0 & 26 \\
        1 & 1 & 0 & 0 & 1 & \textbf{25} \\
        1 & 0 & 1 & 1 & 0 & 22 \\
        1 & 0 & 0 & 1 & 1 & \textbf{19} \\
        1 & 0 & 1 & 0 & 1 & 21 \\
        1 & 1 & 1 & 1 & 1 & \textbf{31} \\
    \end{tabular}
\end{center}
\textbf{Bold} values are prime wrt the operation $\otimes$ given in the lecture.

We get $x^4+x^3+1$, $x^4+x+1$, and $x^4+x^3+x^2+x+1$ as irreducible polynomials.

\subsection*{Exercise 1.2}

Construct $\F_{16}$:\begin{eqnarray*}
    \F_{16} &\cong& F_2[x] / \langle x^4+x^3+x^2+1 \rangle\\
    &\cong& \F_2[x] / \langle x^4+x^3+1 \rangle\\
    &\cong& \F_2[x] / \langle x^4+x+1 \rangle
\end{eqnarray*}

\subsection*{Exercise 1.3}

\begin{eqnarray*}
    m_1 &=& 7_{10} = 111_2 \leadsto x^2+x+1\\
    m_2 &=& (7_{10}, 1_{10}) = (0111_2, 0001_2) \leadsto (x^2+x+1, 1)
\end{eqnarray*}

\section*{Exercise 2}

Let $p$ be a prime.

\subsection*{Exercise 2.1}

\begin{enumerate}[label=\alph*)]
    \item Suppose $p\nmid a$. Prove that $a^{p-1} \equiv 1\mod p$.

        Regard $a$ as element in $\Z_p=\F_p$. Then $a\equiv 0\mod p$, $a\in\Z_p^\times = \F_p^\times$.
        $\F_p^\times$ is a group with $|\F_p^\times| = \varphi(p) = p-1$. Hence $k\cdot ord(a)=p-1$
        for some $k\in\N$.  \[a^{p-1}=a^{k\cdot ord(a)} = \left(a^{ord(a)}\right)^k = 1^k = 1\]

    \item Given $p\mid ab$, show that $p\mid a$ or $p\mid b$.

        Assume $p\nmid a$. Then $\gcd(a,p)=1$ and from Bezout's Lemma we get a representation
        $sp+ta=1,\;s,t\in\Z$.

        Multiplying by $b$ gives $p(sb)+(ab)t\;=\;b$. Since $p\mid ab$, there exists $k\in\Z$ s.t.
        $pk=ab$. Plugging into the first equeation gives $p(sb+kt)=b$.

    \item $(p-1)! \;\equiv\; -1\mod p$ (See book page 80)
\end{enumerate}

\subsection*{Exercise 2.2}

Show: $\Z_N$ is a field $\Leftrightarrow$ $N$ is prime.

``$\Rightarrow$'': $\Z_N$ is a field, then $|\Z_N^\times|=N-1=\varphi(N)$.

Assume $N$ is not prime. Then $N = p_1^{\alpha_1}\cdot\hdots\cdot p_k^{\alpha_k}$ and
\[\varphi(N) \;=\; \prod_{i=1}^k\,\underbrace{p_i^{\alpha_i-1}\cdot(p_i-1)}_{<p_i^\alpha}\;<\;N-1\]

``$\Rightarrow$'': $N$ is prime.

$\Z_n$ is a commutative ring with 1.

It remains to show that $\Z_N \setminus \{0\}$ is a group (equivalently $\Z_N\setminus\{0\}=\Z_N^\times$).

$\varphi(N)= N-1 = |\Z_N|-1$, hence $|Z_N^\times|=N-1=|\Z_N\setminus\{0\}$. Thus $Z_N$ is a field.

\section*{Exercise 3}
\subsection*{Exercise 3.1}

For a multiplicative inverse to exist, $\gcd(x, N)$ must be 1. Under this condition, there exists $u,v$
such that $17\cdot u+22\cdot v=1$.

\begin{itemize}
    \item $13^{-1} \mod 47 = 29$
    \item $17^{-1} \mod 22 = 13$
    \item $6^{-1} \mod 30$: $6\mid 30\Rightarrow 6^{-1}$ doesn't exist.
\end{itemize}

\subsection*{Exercise 3.2}

\begin{enumerate}[label=\alph*)]
    \item
        \begin{eqnarray*}
            && 4x+6=2\mod 7\\
            &\Leftrightarrow& 4x = 2-6 \mod 7\\
            &\Leftrightarrow& 4x = -4\mod 7\\
            &\Leftrightarrow& 4x = 3 \mod 7\quad(*)
        \end{eqnarray*}
        \[\gcd(4,7) = 1 \Rightarrow\exists u,v:4u+7v=1\]
        $4^{-1}=2$, so
        \begin{eqnarray*}
            x&=&c\cdot 4^{-1}\mod 7\\
            &=& 3\cdot 2\mod 7\\
            &=& 6 \mod 7
        \end{eqnarray*}
\end{enumerate}

\section*{Exercise 4}
$\F_2[x]/\langle t^4+t+1\rangle$
\subsection*{Round 1}
$m=(7)=(1+t+t^2)$

$\mathrm{inv}(1+t+t^2)=(t^2+t)=\begin{pmatrix}0&1&1&0\end{pmatrix}^T$

\subsubsection*{GF2 LM}
\[\begin{pmatrix}1&1&1&0\\0&1&1&1\\1&0&1&1\\1&1&0&1\end{pmatrix}
    \begin{pmatrix}0\\1\\1\\0\end{pmatrix}=\begin{pmatrix}0\\0\\1\\1\end{pmatrix}
        = t^3+t^2\]

Sbox const $\rightarrow t^3+t^2+t^2+t=t^3+t$

$(t^3+t)$ ND SR ND MC

\end{document}

